% -*- root: ../Technical_report.tex -*-
\section{Ideas about new development}

\subsection{Roles of the Application}

We define the user as the reporter of the incidence and the maintenance personnel as the people in charge of fixing the faults.

Every one uses the same application, but it will be shown a different menu depending on the role after the authentication.

\begin{itemize}
\item \textbf{Admin role: } The person or people in charge of banning and modifying manually what was automatically set by the system.

\item\textbf{Maintenance person:} You can see as a Maintenance person what task has been assigned to you and you can mark them as finished or as a false fault.

You can also ask for more information about the fault if it's not precise enough.

\item \textbf{User role: }  We may define subroles such as professor, student or workers (waiters, secretaries ... ) because of the liability of each group of people.

\end{itemize}

Everyone is a user and everyone can report faults but not  every one can be Maintenance person or user.

\subsection{Task priority}
We talked about 2 options:

\begin{itemize}
\item The user sets the priority of the task and the system has to take on account the liability of the user to set a real priority to the task.

\item There are tags and categories the user can add to the report and the system computes a real priority to the task.

Within this idea we need to add \textit{others} category, just in case.
\end{itemize}

\subsection{Location}

We have to distinguish 2 cases, if you are inside a building or not.

If you are not inside a building, the location will be automatic by GPS technology, but when faults are relative to inside of buildings material it will be necessary to ask the user the exact location. This will be ask by an interactive map.


We will also use WiFi location because sometimes GPS is a slow technology to be set and reliable.

\subsection{Manage task assignation}
The system automatically assigns tasks, depending on the location of the faults and the Maintenance person and the availability of the maintenance personnel.

Admin can set manually the task assignation.

We should take care of the following situation:

A Maintenance person works a lot and fix every task he was assigned, and, because he is free, the system assigns some more. On the other side, a Maintenance person is not working. As he has lot of pending task, system will not add more task to him. 

Plus, we can't just count the number of faults fixed, because there are ones more difficult than others. May be we should add \textit{difficulty of fault} depending on category to try control the how much job you have been assigned.

\subsection{Graphical interface}
We have Guille's application in which we will add just a profile page in which users can see
\begin{itemize}
\item the false faults they have reported .
\item the fault's reported state.
\end{itemize}

Every user's profile will be visible to the admin but he won't be able to see any personal data.

The admin can see profiles so before he can ban an user, he can check how many false and repeated faults he has reported.


\paragraph{Maintenance person interface}
The interface will show by color code the faults assignments classifying  they by pending, solved or not started.


\subsection{Repeated faults}
When an user is going to report a fault, the system will suggest if it's repeated showing possible faults. If the users says it's a new fault, it will be still marked as a possible repeated fault.

When a task is assigned to a Maintenance person, all possible repeated faults will be assigned to the same Maintenance person. It's Maintenance person's responsibility to say if they are the same or not.

If they are not the same, the task will be enqueued and the system will automatically be assigned to a Maintenance person, maybe the himself or may be not.


\subsection{Communication}

When a fault is fix, the system will notify the user who reported it.

\section{Questions to ask}

\begin{itemize}
\item When an emergency occurs in UAM (e.g., a fire, a flood...), does everyone get an alert?
\item Gamified experience? Do we ``reward'' the user for reporting faults or shall we instead offer detailed reports so the IT department can reward users as they want? (Credit's ETCS reward is possible, in the same way as languages)
\item We should take care of the following situation:

A Maintenance person works a lot and fix every task he was assigned, and, because he is free, the system assigns some more. On the other side, a Maintenance person is not working. As he has lot of pending task, system will not add more task to him. 

If he can assure us every one will work normally, that will solve the problem.
\end{itemize}

