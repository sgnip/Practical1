% -*- root: ../Technical_report.tex -*-
\section{Abstract}

\textit{
In order to improve their performance, more and more maintenance services in charge of different entities or organizations have started to automate their labour using technologies and software systems specially designed taking into account the needs this sector has.\\
Among these needs we can find the lack or poor coordination between members or teams of the maintenance staff, the manual assignment of repair tasks and the great waste of time and personnel resources scheduling revisions of the facilities in order to find faults that may have appeared (revisions that usually are fruitless and do not detect all the faults in time, remaining unrepaired for even days).\\
In this paper, we present an project proposal of a web application tailored to needs of the UAM. The UAM has detected some potential maintenance problems which include the ones mentioned above.\\
Our main contribution relates to the development of an application, Fault Manager Lite, whose functionality is gathered in this document. We state that our application, developed after an exhaustive proccess of investigation and analysis, is able to get rid of any of the problems that the maintenance service of the UAM suffers from.\\
}