% -*- root: ../Technical_report.tex -*-
\chapter{Catalog of Requirements}
\label{chapRequirements}
\section{Roles}
It is important to define which roles exists in FML.

\paragraph{Reporter role} \label{ReporterRole} This is the role applied to anyone who want to report some fault. It will be taken into account if the person is a student, a PhD or a teacher. 

\paragraph{Maintenance Personnel} \label{MaintenancePersonnel}

This are the roles applied to people in charge of maintenance. There will be 2 subcategories inside of Maintenance Personnel. 

\begin{itemize}
\item The people in charge of each department.
\item The maintenance personnel responsible to fix the faults.
\end{itemize}

As there are 8 departments, there will be 8 categories. Each maintenance person should be tagged in, at least, 1 department. It is also compatible being in charge of a department and being responsible to fix faults.

\subparagraph{Exceptions} The \textbf{cleaning department} only needs a person in charge of the whole department and it is his responsibility to assign faults to their employees and to mark faults as solved. 

This department is also special because there is a different department in each faculty, so the must be at least 1 person in charge of each faculty cleaning department.

\paragraph{Admin Role} Is the person (or group of people) in charge of UAM's maintenance system in general. All admins will have permissions to change everything, as they are the bosses of everything related with maintenance.

\section{Functional requirements}

\subsection{Task Manager subsystem}


\subparagraph{Faults definition: } Each fault will have 3 possible states: \textit{Pending to assign, assigned, solved}. 

If a fault is impossible to be fixed, it will correspond to the administrator to take care personally of it.


\subparagraph{Categorizing faults: } There should be categories to tag each fault depending on the estimated difficulty (numeric value between 1 and 5), the urgency (urgent or not) and the department who needs to take care of it.


\subparagraph{Fault's assignments: } FML will assign to maintenance personnel faults to be fixed (depending on the fault's category and maintenance person load of work and capability to solve the fault)

This will be set automatically and the system's administrators will be able to reassign the faults as they pleased.

The person in charge of a department will also be able to reassign faults that have been assigned to his department or to someone inside his department to his employees.

As we mentioned before, the cleaning departments are special, because all cleaning faults in a building will be assign to the person in charge of that building's cleaning.


\subparagraph{Task managing: } Everyone from the maintenance personnel will have read access to the faults data base through a Graphical User Interface (GUI). 

They will just have write access to faults they have been assigned to. \footnote{Write access is needed so the maintenance person change fault's state to \textit{solved} or to \textit{in progress}}

Admin will have write access to all faults in the system, and the person in charge of a department will have write access to all faults assigned to someone inside his department.


\paragraph{When a fault is fixed: } The maintenance person assigned (or the manager of that department) will mark as solved the fault. At that time, the reporter will get notify and thanked. 


\subsection{Report subsystem}

\paragraph{Reporting a fault: } 


\subparagraph{} Reporting a fault with the following information: \textit{Location, photograph (optional) and description}

\subparagraph{} See and check reported faults and see its state.

\subparagraph{} Answer questions the maintenance personnel will ask about the reported fault.



\paragraph{Duplicated faults: } This is a big problem to be solved. What we propose to solve it has 2 aspects.
\begin{itemize}
\item FML will be able to mark as \textit{possible duplicated}, so all possible duplicates will be assign to the same maintenance people, so he can verify if they are duplicated or not.

\item On the other side, to prevent duplicated faults reports, FML will show a message showing possible duplicated faults to the reporter. If the reporter marks the fault he is reporting as duplicated, he will earn some points too and will be added into the list of people to be notified when the task is solved.
\end{itemize}


\subsection{Communication subsystem - Notifications and messaging}

\subparagraph{Communication: } Maintenance person assigned to fix a fault should be able to ask the fault's reporter for more information.


\subsection{Users and Profile manager system}

\paragraph{Log in: } Everyone must be able to login using UAM credentials. We will use the service provide by UAM to authenticate an user.

If the user is logging in on a smartphone, the system shouldn't ask for the password more than once (except for some specific operations defined in the role-depending requirements, user role category (\ref{Specifics_Secure_Requirements_for_user}))

\paragraph{Updating profile: } What if a student become a teacher and then his email needs to be changed? To fix this, every user must be able to update his profile's information. 


\subsection{Faults History}

\paragraph{Seeing history} As we think transparency is really important, every reporter will be able to see all history of faults. In non-functional requirements we define how this should be done. 

\paragraph{Statistics: } In the profile page each user should be able to see the faults he has reported and it's state.

Plus, he should be able to see his conversations with maintenance personnel.

%%%%%%%%%%%%%%%%%%%%%%%%%%%%%%%%%%%%%%%%%%%%%%%%%%%%%%%%%%%%%%%%%%%%%%%%%%%%%%%%%%%%%%%%%%%%%%%%%%%%%%%%%%%%%%%%%%%%%%%%%%%%%%%%%%%%%%%%%%%%%%%%%%%%%%%%%%%%%%%%%%%%%%%%%%%%%%%%%%%%%%%%%%%%%%%%%%%%%%%%%%%%%%%%%%%%%%%%%%%%%%%%%%%%%%%%%%%%%%%%%%%%%%%%%%%%%%%%%%%%%%%%%%%%%%%%%%%%%%%%%%%%%%%%%%%%%%%%%%%%%%%%%%%%%%%%%%%%%%%%%%%%%%%%%%%%%%%%%%%%%%%%%%%%%%%%%%%%%%%%%%%%%%%%%%%%%%%

\section{Non-Functional}

\subsubsection{General Requirements}

This requirements are defined for all roles inside FML (Admin, maintenance personnel and user)

\paragraph{Profile: } Everyone should have a profile with some private information such as name or email.

\paragraph{Lightweight application: } 
% Revisar esta frase, no me convence la sintaxis
FML will be lightweight enough to be ran in  4 years old Chrome, Firefox, IE's versions. 

\paragraph{Read Access to task: } Every FML's user will be able to see all reported faults (we think transparency is really important). This will be shown on a map of the UAM. Each fault will appear as an arrow pointing to fault's location and color-tagged depending on its state (solved, pending, assigned).

\subsubsection{Role-depending requirements}

\paragraph{Reporter: } FML should provide users the following abilities:
\begin{itemize}
\item The specifics requirements where re-authentication is needed are: \textit{Answering questions asked by maintenance personnel}   \label{Specifics_Secure_Requirements_for_user} and \textit{updating profile information}.
\item The interface to report a fault should be easy enough to let the user report the fault without losing much time.
\item The amount of points earned should be visible and the information of how much points does the user need to earn ECTS credit should be visible inside FML.

\end{itemize}
