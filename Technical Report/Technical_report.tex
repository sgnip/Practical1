\documentclass{report}

\usepackage{fancyhdr} % Cabeceras de página
\usepackage{lastpage} % Módulo para añadir una referencia a la última página
\usepackage{titling} % No tengo claro para qué es esto
\usepackage[left=2cm,right=2cm,top=3cm,bottom=2cm]{geometry} % Márgenes
\usepackage[T1]{fontenc}
\usepackage[utf8x]{inputenc}
\usepackage{xspace}
\usepackage{graphicx}
\usepackage{tikz}
\usepackage{wrapfig}
\usepackage{hyperref}

\hypersetup{
  hyperindex,
    colorlinks,
    allcolors=blue!60!black
}


\setcounter{secnumdepth}{3}


\title{Fault Manager Lite (Technical report)}
\date{\today}
\author{SGNIP company}

\fancyhf{}
\fancypagestyle{plain}{%
	\lhead{\small \itshape \thetitle\, -\, \thedate\, -\, PINGS}
	\rhead{\vspace{-20pt} \includegraphics[width =40 pt]{../Logo.jpg}}
	\cfoot{\thepage\ of \pageref{LastPage}}
	\rfoot{}
}

\begin{document}
\maketitle
\tableofcontents
\newpage
\begin{abstract}
\end{abstract}

\chapter{Introduction}

The Autonomous University of Madrid (UAM) has reported the numerous problems it has on detecting faults that arise on its campus and its facilities, whose reparation usually takes excessive time and is poorly organised. A late detection of faults in the facilities delays its reparation, stopping its users to continue using them as normal and complicating the maintenance staff labour. Regarding the wishes of the campus users and realising the maintenance problems it has, the UAM has organised a contest to choose the best project proposal that solves them.
This is where our organization, Triforce, enters the scene: we have analyzed the problem exhaustively and designed a web application, Fault Manager Lite (FML), that meets all the requirements expected, solves the problems and also includes new extra features that makes it even more useful.

\paragraph{Purpose} The purpose of this Software Requirements Specification (SRS) document is to provide a detailed description of the functionalities of the FML system. This document will cover each of the system's intended features, as well as offer a preliminary glimpse of the software application's User Interface (UI).

\section{Structure}

\section{Definitions and abbreviations}

\begin{itemize}
\item \textbf{SRS: } Software Requirements Specification.

\item \textbf{FML: } Fault Manager Lite. This is the System's name.

\end{itemize}


\section{Methodology}

The methodological procedure followed for preparing the SRS of the software system FML we have developed includes the following techniques:
\begin{enumerate}
\item Deep analysis of the information given by the UAM: the potential maintenance problem it has, the causes that led to this problem and various aspects we must consider before proceeding to the next step (web application that must work on PCs, tablets and smartphones; target users; basic purpose).
\item Application of the technique of brainstorming to generate ideas for the application. Brainstorming was then applied to these general ideas in order to give shape to them. The results of the brainstorm carried out are shown in the Annex A. This step resulted on the detection of 5 subsystems.
\item Research on the Internet about other similar systems, analyzing the functionality of these systems and describing them in a structured way by specifying their advantages and disadvantages, and extracting good ideas to incorporate to our own project, increasing its market value. These ideas also helped putting the final touches to our ideas from the brainstorm.
\item Interview with one member of the UAM's technical staff. In this interview we clarified the obscure ideas we still had, and modified some of our previous ideas to adapt them to the answers given. The answers to the interview were well-considered; as the technical staff will be the main user of our application, their point of view is important to us.
\item Departing from the ideas obtained, defined the requirements (functional and non-functional) of the application and design attractive mock-ups that fulfill them. These mock-ups are only a demo and may change in the final version of the application.
\end{enumerate}

\chapter{Project Definition}

FML is based on the jolly cooperation between the users of the facilities of the UAM campus and the maintenance staff in charge of them.

As the users will be the ones that will detect the sooner faults on the facilities the campus offers them, they are also the most suitable to report the problems they are having, in order to getting them fixed as soon as possible. With the FML web application, it will only take less than a minute to fill the form and send it to the maintenance staff.

Using the reports of faults detected in the campus, the maintenance personnel will stop losing its precious time revising the installations looking for faults and will be able to focus on the repairs. Apart from this benefit, the maintenance staff will also have a better way to coordinate efforts, as the FML will provide an automatic assignment system to assign repairs to each of the members avoiding overloading any of them and taking into account their distance to the problem, saving time on displacements.

In summary, the users will be able to easily report faults on facilities they are using in order to have them fixed in the less time possible, while the maintenance personnel will multiply its current performance as the majority of their resources will stop being wasted on revisions, but on repairs. We state that the FML web application is the answer to the UAM problems, as we will show you with this SRS.


\section{Goals and functionalities}

Our main goal is the development of an extremely functional, bug-free web application, that runs either on PCs, tablets or smartphones and facilitates the labor of the maintenance staff of the UAM.

Based on the jolly cooperation between the members of the UAM community (students, teaching and research staff and administration and services personnel) and the maintenance personnel of the facilities and installations of the UAM campus, this application has been designed to solved the potential maintenance problems that the UAM has declared to have.

The problems we aim to alleviate with this application include the following:


\begin{itemize}
\item Difficulty of detecting faults on facilities of the campus.
\item Late detection of the problems, which implies repairs are not done on time.
\item Excessive resources and time wasted on revisions looking for potential faults.
\item Bad coordination between members of the technical staff.
\item Frequent overloading of some repairmen because of bad coordination.
\item Inability of a repairman to instantly inform that a fault has just been solved.
\item Difficulty of the users to report faults to the maintenance services.
The report system lacks of mobile functionality.
\item Lack of real-time visualization of pending and finished repairs.
\item Lack of real-time visualization of assigned tasks.
\item No faults history.
\item No statistics of faults.
\end{itemize}

\section{Subsystems or Modules}

The FML system has been designed to resolve these problems making use of a user-friendly interface, avoiding unnecessary or distracting buttons or effects. This system has been divided in several modules listed below, each of those offering one specific functionality and, in conjunction, solving the problems mentioned above:
\begin{itemize}
\item Task Manager.
\item Report System.
\item Notifications and Messaging System.
\item Users and Profile Managers.
\item Faults History.
\end{itemize}

There are several applications 'on the market' which are similar to our proposed app in some way or another and they could represent real and competitive alternatives to ours. However, none of these web applications can offer all the features FML offers (furthermore, we plan on adding some features which are not currently available in any of these applications), so our system is indeed the best solution proposed to solve the problems the UAM maintenance staff has to deal with. After our research of the Internet, a comparison of these competitors' applications is gathered in Annex B.

\chapter{Catalog of Requirements}

\section{Document Service Requirements}

\subsection{Functional}

%%%%% Order requirements for subsystems defined above.

\subsubsection{General Requirements}

This requirements are defined for all roles inside FML (Admin, maintenance personnel on both levels and reporters)

\subsubsection{Requirements about roles inside FML}

We need to define 3 roles.

\subsubsection{Reporter role} This is the role applied to anyone who want to report some fault. It will be taken into account if the person is a student, a PhD or a teacher.

\subsubsection{Maintenance Personnel}

This are the roles applied to people in charge of maintenance. There will be 2 subcategories inside of Maintenance Personnel.

\begin{itemize}
\item The people in charge of each department.
\item The maintenance personnel responsible to fix the faults.
\end{itemize}

As there are 8 departments, there will be 8 categories. Each maintenance person should be tagged in, at least, 1 department. It is also compatible being in charge of a department and being responsible to fix faults.

\paragraph{Exceptions} The \textbf{cleaning department} only needs a person in charge of the hole department and it is his responsibility to assign faults to their employees and to mark faults as solved.

This department is also special because there is a different department in each faculty, so the must be at least 1 person in charge of each faculty cleaning department.


\paragraph{Requirements focused on users}


\subparagraph{Log in} Everyone must be able to login using UAM credentials. We will use the service provide by UAM to authenticate an user.

If the user is login in a smartphone, the system shouldn't ask for the password more than once (except for some specific operations defined in the role-depending requirements, user role category (\ref{Specifics_Secure_Requirements_for_user}))

\subparagraph{Lightweight application}
FML will be lightweight so it will run in old versions (we will support releases made up to four years before FML release date) of Chrome, Firefox and Internet Explorer.

\subparagraph{Updating profile} What if a student become a teacher and then his email needs to be changed? To fix this, every user must be able to update his profile's information.

\subparagraph{Faults definition} Each fault will have 3 possible states: \textit{Pending to assign, assigned, solved}. If a fault can't be fixed, the administrator will take care of it.

\paragraph{Requirements focused on maintenance personnel}

\subparagraph{Categorizing faults} There should be categories to tag each fault depending on the estimated difficulty (numeric value between 1 and 5), the urgency (urgent or not) and the department who needs to take care of it.

\subparagraph{Fault's assignments} FML will assign to maintenance personnel faults to be fixed (depending on the fault's category and maintenance person load of work and capability to solve the fault)

This will be set automatically and the system's administrators will be able to reassign the faults as they pleased.

The person in charge of a department will also be able to reassign faults that have been assigned to his department or to someone inside his department to his employees.

As we mentioned before, the cleanings departments are special, because all cleaning faults in a building will be assign to the person in charge of that building's cleaning.

\subparagraph{Task managing} Everyone from the maintenance personnel will have read access to the faults data base through a Graphical User Interface (GUI).

They will just have write access to faults they have been assigned to.\footnote{Write access is needed so the maintenance person change fault's state to \textit{solved} or to \textit{in progress}.}

The administrator will have write access to all faults in the system, and the person in charge of a department will have write access to all faults assigned to someone inside his department.

\subparagraph{Communication} Maintenance person assigned to fix a fault should be able to ask the fault's reporter for more information.

\subparagraph{When a fault is fixed} The maintenance person assigned (or the manager of that department) will mark as solved the fault. At that time, the reporter will get notify and thanked.

\subparagraph{Duplicated faults} This is a big problem to be solved. What we propose to solve it has 2 aspects.
\begin{itemize}
\item FML will be able to mark as \textit{possible duplicate}, so all possible duplicates will be assign to the same maintenance person, so he can verify if they are actually duplicated.

\item On the other side, to prevent duplicated faults reports, FML will show a message showing possible duplicates faults to the reporter. If the reporter marks the fault he is  as duplicated, he will earn some points too and will be added into the list of people to be notified when the task is solved.
\end{itemize}

\subsubsection{Role-depending requirements}

\paragraph{Reporter} FML should provide users the following abilities:
\begin{itemize}
\item Reporting a fault with the following information: \textit{Location, photograph (optional) and description}
\item See and check reported faults and see its state.
\item Answer questions the maintenance personnel will ask about the reported fault.
\item The specifics requirements where re-authentication is needed are: \textit{Answering questions asked by maintenance personnel}   \label{Specifics_Secure_Requirements_for_user}
\end{itemize}


\subsection{Non-Functional}

\subsubsection{General Requirements}

This requirements are defined for all roles inside FML (Admin, maintenance personnel and user)

\paragraph{Profile} Everyone should have a profile with some private information such us name.

\subparagraph{Statistics} In the profile page each user should be able to see the faults he has reported and it's state.

Plus, he should be able to see his conversations with maintenance personnel.


\subsubsection{Role-depending requirements}

\paragraph{User Role}
FML should provide users the following abilities:
\begin{itemize}
\item The interface to report a fault should be easy enough to let the user report the fault without losing much time.
\item The amount of points earned should be visible and the information of how much points does the user need to earn ETCS credit should be visible inside FML.
\end{itemize}


\chapter{Design and Mock-ups}

\begin{figure}[hbtp]
\centering
\begin{minipage}{0.3\textwidth}
\includegraphics[width=\textwidth]{img/MainPage.png}
\end{minipage}
\hspace{0.1\textwidth}
\begin{minipage}{0.3\textwidth}
\includegraphics[width=\textwidth]{img/Settings.png}
\end{minipage}
\caption{Main page and settings page for Fault Manager Lite}
\label{imgMainPage}
\end{figure}!

We show some mock-ups of the application, specifically designed to show the flow in certain use cases.

\section{Reporting a fault}

From the main screen (figure \ref{imgMainPage}), when the user presses the button ``Report'', he will be able to choose a category for the fault, then will fill the details and then get a page acknowledging the report and showing the estimated response time (figure \ref{imgReportFlow}).

\begin{figure}[hbtp]
\centering
\begin{minipage}{0.3\textwidth}
\includegraphics[width=\textwidth]{img/Categories.png}
\end{minipage}
\hspace{0.02\textwidth}
\begin{minipage}{0.3\textwidth}
\includegraphics[width=\textwidth]{img/ReportPage.png}
\end{minipage}
\hspace{0.02\textwidth}
\begin{minipage}{0.3\textwidth}
\includegraphics[width=\textwidth]{img/ReportReceived.png}
\end{minipage}
\caption{Basic flow for reporting a fault.}
\label{imgReportFlow}.
\end{figure}

For the location, we will show the user a map and also a list with all the possible options (figure \ref{imgLocation}). The user will be able to select specific places (for example, a certain lab or classroom) or more generic (e.g., third floor or the whole building). If localization APIs are available on the device (either location by AGPS or WiFi), the application will use them to show the user the best location option.


\begin{figure}[hbtp]
\centering
\begin{minipage}{0.3\textwidth}
\includegraphics[width=\textwidth]{img/Map.png}
\end{minipage}
\hspace{0.02\textwidth}
\begin{minipage}{0.3\textwidth}
\includegraphics[width=\textwidth]{img/Location.png}
\end{minipage}
\caption{Location screens for the application}
\label{imgLocation}
\end{figure}

\section{Fault tracking}

\begin{figure}[hbtp]
\centering
\begin{minipage}{0.3\textwidth}
\includegraphics[width=\textwidth]{img/FaultLog.png}
\end{minipage}
\hspace{0.02\textwidth}
\begin{minipage}{0.3\textwidth}
\includegraphics[width=\textwidth]{img/Messaging.png}
\end{minipage}
\caption{Fault log and messaging screen for the application.}
\label{imgTracking}
\end{figure}

The user will be able to see the faults he's been subscribed too, either because he reported them or because he marked them as duplicated. The figure \ref{imgTracking} shows how can he see all his faults and the chat screen that will show up if the technicians need to clarify details of the report.

\chapter{Conclusions}


\appendix

\chapter{Brainstorming}

% -*- root: ../Technical_report.tex -*-
\section{Ideas about new development}

\subsection{Roles of the Application}

We define the user as the reporter of the incidence and the maintenance personnel as the people in charge of fixing the faults.

Every one uses the same application, but it will be shown a different menu depending on the role after the authentication.

\begin{itemize}
\item \textbf{Admin role: } The person or people in charge of banning and modifying manually what was automatically set by the system.

\item\textbf{Maintenance person:} You can see as a Maintenance person what task has been assigned to you and you can mark them as finished or as a false fault.

You can also ask for more information about the fault if it's not precise enough.

\item \textbf{User role: }  We may define subroles such as professor, student or workers (waiters, secretaries ... ) because of the liability of each group of people.

\end{itemize}

Everyone is a user and everyone can report faults but not  every one can be Maintenance person or user.

\subsection{Task priority}
We talked about 2 options:

\begin{itemize}
\item The user sets the priority of the task and the system has to take on account the liability of the user to set a real priority to the task.

\item There are tags and categories the user can add to the report and the system computes a real priority to the task.

Within this idea we need to add \textit{others} category, just in case.
\end{itemize}

\subsection{Location}

We have to distinguish 2 cases, if you are inside a building or not.

If you are not inside a building, the location will be automatic by GPS technology, but when faults are relative to inside of buildings material it will be necessary to ask the user the exact location. This will be ask by an interactive map.


We will also use WiFi location because sometimes GPS is a slow technology to be set and reliable.

\subsection{Manage task assignation}
The system automatically assigns tasks, depending on the location of the faults and the Maintenance person and the availability of the maintenance personnel.

Admin can set manually the task assignation.

We should take care of the following situation:

A Maintenance person works a lot and fix every task he was assigned, and, because he is free, the system assigns some more. On the other side, a Maintenance person is not working. As he has lot of pending task, system will not add more task to him. 

Plus, we can't just count the number of faults fixed, because there are ones more difficult than others. May be we should add \textit{difficulty of fault} depending on category to try control the how much job you have been assigned.

\subsection{Graphical interface}
We have Guille's application in which we will add just a profile page in which users can see
\begin{itemize}
\item the false faults they have reported .
\item the fault's reported state.
\end{itemize}

Every user's profile will be visible to the admin but he won't be able to see any personal data.

The admin can see profiles so before he can ban an user, he can check how many false and repeated faults he has reported.


\paragraph{Maintenance person interface}
The interface will show by color code the faults assignments classifying  they by pending, solved or not started.


\subsection{Repeated faults}
When an user is going to report a fault, the system will suggest if it's repeated showing possible faults. If the users says it's a new fault, it will be still marked as a possible repeated fault.

When a task is assigned to a Maintenance person, all possible repeated faults will be assigned to the same Maintenance person. It's Maintenance person's responsibility to say if they are the same or not.

If they are not the same, the task will be enqueued and the system will automatically be assigned to a Maintenance person, maybe the himself or may be not.


\subsection{Communication}

When a fault is fix, the system will notify the user who reported it.

\section{Questions to ask}

\begin{itemize}
\item When an emergency occurs in UAM (e.g., a fire, a flood...), does everyone get an alert?
\item Gamified experience? Do we ``reward'' the user for reporting faults or shall we instead offer detailed reports so the IT department can reward users as they want? (Credit's ETCS reward is possible, in the same way as languages)
\item We should take care of the following situation:

A Maintenance person works a lot and fix every task he was assigned, and, because he is free, the system assigns some more. On the other side, a Maintenance person is not working. As he has lot of pending task, system will not add more task to him. 

If he can assure us every one will work normally, that will solve the problem.
\end{itemize}




\chapter{Evaluate current technology}
% -*- root: ../Technical_report.tex -*-
\section{Suggested webpages to look at}

Ivan's fault if this is empty xD

\section{Our own webpages to look at}

\subsection{Wunderlist}

\begin{itemize}
\item Nice interface to take into account for the task management.
\item Maybe shared list's so every handyman in science building can see pending task of others. (to ask)
\item See completed faults you fix. (pretty obvious).
\item Dead line for each fault, automatically set by the system taking into account the priority and the category. (to ask)
\end{itemize}


\subsection{Apple watch/YoApp/TapToTalk}

\begin{itemize}
\item Add some sort of really fast communication between handymen, so they don't have to call one another.
\item Add few types of default messages like “I'm going” or “Impossible” or “Need help”
\item Things to think about:
\begin{itemize}\label{AppleWatch}
\item What about broadcasting messages to all handyman? Or may be just to a few of them? Maybe making groups and you send notification to a group.
\item May be we should add some not so fast but more efficient communication way for handyman.
\end{itemize}
\end{itemize}

\subsection{Trello}

\begin{itemize}
\item Public taskboard that anyone can see? Only registered members?
\item Add one more state to task (pending, working, finished) better than wunderlist system (done or todo).
\end{itemize}

\subsection{Asana}
\begin{itemize}
 \item  Dead line for tasks in days or even in hours?
\end{itemize}

\chapter{Meetings}

% -*- root: ../Technical_report.tex -*-
\section{Meeting Announcements}

\subsection{Meeting Announcement - 1}

\textbf{From: } Guillermo Julián Moreno\\
\textbf{To: } Iván Márquez Pardo, Víctor de Juan Sanz.\\

\textbf{Date and Time: } 26/01/2015, 14:00.\\
\textbf{Place: } third floor of the EPS Library.\\

\textbf{Purpose} First meeting as a team. Start working on the project.\\

\textbf{1- Agenda:}
\begin{enumerate}
\item Overview of the project assigned.
\item Divide work and start the project (if possible).
\end{enumerate}
\textbf{2- Decision Follow-Up:} None.\\
\textbf{3- Documentation:}
\begin{enumerate}
\item Statement/definition of the project.
\end{enumerate}


\subsection{Meeting Announcement - 2}

\textbf{From: } Víctor de Juan Sanz\\
\textbf{To: } Iván Márquez Pardo, Guillermo Julián Moreno.\\

\textbf{Date and Time: } 28/01/2015, 14:00.\\
\textbf{Place: } third floor of the EPS Library.\\

\textbf{Purpose} Brainstorming session.\\

\textbf{1- Agenda:}
\begin{enumerate}
\item Brainstorming session.
\item Gather ideas in a draft paper.
\end{enumerate}
\textbf{2- Decision Follow-Up:} 
\begin{enumerate}
\item The three of us have read the whole documentation.
\item We have to come to this meeting with some ideas for the project.
\end{enumerate}
\textbf{3- Documentation:}
\begin{enumerate}
\item Statement/definition of the project.
\end{enumerate}


\subsection{Meeting Announcement - 3}

\textbf{From: } Iván Márquez Pardo.\\
\textbf{To: } Víctor de Juan Sanz, Guillermo Julián Moreno.\\

\textbf{Date and Time: } 29/01/2015, 11:45.\\
\textbf{Place: } Science building, module 11.\\

\textbf{Purpose} Brainstorming session.\\

\textbf{1- Agenda:}
\begin{enumerate}
\item Second (and final) brainstorming session.
\item Gather ideas in a draft paper.
\end{enumerate}
\textbf{2- Decision Follow-Up:} 
\begin{enumerate}
\item The three of us have read the whole documentation.
\item We have to come to this meeting with some ideas for the project.
\end{enumerate}
\textbf{3- Documentation:}
\begin{enumerate}
\item Brainstorming draft paper.
\end{enumerate}


\subsection{Meeting Announcement - 4}

\textbf{From: } Guillermo Julián Moreno\\
\textbf{To: } Iván Márquez Pardo, Víctor de Juan Sanz.\\

\textbf{Date and Time: } 02/02/2015, 19:00.\\
\textbf{Place: } third floor of the EPS Library.\\

\textbf{Purpose} Review of work. \\

\textbf{1- Agenda:}
\begin{enumerate}
\item Review of work already done.
\item Assignment of new tasks for the week.
\end{enumerate}
\textbf{2- Decision Follow-Up:} 
\begin{enumerate}
\item The three of us have completed their previous tasks.
\end{enumerate}
\textbf{3- Documentation:}
\begin{enumerate}
\item Statement/definition of the project.
\end{enumerate}


\subsection{Meeting Announcement - 5}

\textbf{From: } Víctor de Juan Sanz.\\
\textbf{To: } Iván Márquez Pardo, Guillermo Julián Moreno.\\

\textbf{Date and Time: } 09/02/2015, 14:00.\\
\textbf{Place: } third floor of the EPS Library.\\

\textbf{Purpose} Preparation of the interview. \\

\textbf{1- Agenda:}
\begin{enumerate}
\item Revision of subsystems.
\item Preparation of the interview with the maintenance manager.
\end{enumerate}
\textbf{2- Decision Follow-Up:} 
\begin{enumerate}
\item We still have to approve the subsystems which form the application.
\end{enumerate}
\textbf{3- Documentation:}
\begin{enumerate}
\item Parts of the project already done, which would be analyzed to clarify obscure parts in the interview.
\end{enumerate}

\subsection{Meeting Announcement - 6}

\textbf{From: } Iván Márquez Pardo.\\
\textbf{To: } Víctor de Juan Sanz, Guillermo Julián Moreno.\\

\textbf{Date and Time: } 11/02/2015, 11:35.\\
\textbf{Place: } Science building, module 11.\\

\textbf{Purpose} Overview of the interview. \\

\textbf{1- Agenda:}
\begin{enumerate}
\item Overview of the interview.
\item Clarify some aspects of the interview.
\item Correction of some parts of the project.
\item Review of the presentation.
\item Practice the oral presentation.
\end{enumerate}
\textbf{2- Decision Follow-Up:} 
\begin{enumerate}
\item Subsystems are now fixed.
\end{enumerate}
\textbf{3- Documentation:}
\begin{enumerate}
\item Interview craft paper.
\end{enumerate}


\section{Meeting Minutes}



\subsection{Meeting Minutes - 1}
\textbf{Date and Time:} 26/01/2015, 14:00. 
\textbf{Length:} 28 minutes. 
\textbf{Participants: } Guillermo Julián Moreno, Víctor de Juan Sanz, Iván Márquez Pardo.

\textbf{Topics: } 
\begin{enumerate}
\item Overview of the project assigned.
\item Divide work and start the project (if possible).
\end{enumerate}

\textbf{Decisions Made:}\\
\begin{enumerate}
\item The next meeting will consist in a brainstorming session.
\end{enumerate}

\begin{tabular}{|p{5cm} c|p{5cm}|p{5cm}|}
\hline Activity & Person Responsible & DeadLine \\\hline
Read the whole documentation given in order to comprehend better the main problem we have to solve and think of general ideas for our project proposal & Víctor de Juan & 29/01/2015\\\hline

Read the whole documentation given in order to comprehend better the main problem we have to solve and think of general ideas for our project proposal & Guillermo Julián Moreno & 29/01/2015\\\hline

Read the whole documentation given in order to comprehend better the main problem we have to solve and think of general ideas for our project proposal & Iván Márquez Pardo & 29/01/2015\\\hline
\end{tabular}



\subsection{Meeting Minutes - 2}
\textbf{Date and Time:} 28/01/2015, 14:00. 
\textbf{Length:} 1 hour. 
\textbf{Participants: } Guillermo Julián Moreno, Víctor de Juan Sanz, Iván Márquez Pardo.

\textbf{Topics: } 
\begin{enumerate}
\item Brainstorming session.
\item Gather ideas in a draft paper.
\end{enumerate}

\textbf{Decisions Made:}\\
\begin{enumerate}
\item We could not finish the brainstorm, so a second brainstorming session is needed.
\end{enumerate}

\begin{tabular}{|p{5cm} c|p{5cm}|p{5cm}|}
\hline Activity & Person Responsible & DeadLine \\\hline
Revision of the current state of the brainstorm & Víctor de Juan & 29/01/2015\\\hline

Look for contradictions (if any) between the ideas from the brainstorm and the project definition & Guillermo Julián Moreno & 29/01/2015\\\hline

Think of more original ideas to add in the brainstorming final session & Iván Márquez Pardo & 29/01/2015\\\hline
\end{tabular}


\subsection{Meeting Minutes - 3}
\textbf{Date and Time:} 29/01/2015, 14:00. 
\textbf{Length:} 47 minutes. 
\textbf{Participants: } Guillermo Julián Moreno, Víctor de Juan Sanz, Iván Márquez Pardo.

\textbf{Topics: } 
\begin{enumerate}
\item Second (and final) brainstorming session.
\item Add new ideas to the draft paper.
\item Start with other parts of the project.
\end{enumerate}

\textbf{Decisions Made:}\\
\begin{enumerate}
\item Revise ideas and finish the brainstorming paper.
\item Start with other parts of the project.
\end{enumerate}

\begin{tabular}{|p{5cm} c|p{5cm}|p{5cm}|}
\hline Activity & Person Responsible & DeadLine \\\hline
Write down clearly in Latex format all the ideas obtained from the brainstorming session & Víctor de Juan & 02/02/2015\\\hline

Start thinking about the design of the application & Guillermo Julián Moreno & 02/02/2015\\\hline

Write down the introduction of the project & Iván Márquez Pardo & 02/02/2015\\\hline
\end{tabular}


\subsection{Meeting Minutes - 4}
\textbf{Date and Time:} 02/02/2015, 19:00. 
\textbf{Length:} 15 minutes. 
\textbf{Participants: } Guillermo Julián Moreno, Víctor de Juan Sanz, Iván Márquez Pardo.

\textbf{Topics: } 
\begin{enumerate}
\item Review of work already done.
\item Assignment of new tasks for the week.
\end{enumerate}

\textbf{Decisions Made:}\\
\begin{enumerate}
\item Corrected some obscure parts of the work done.
\item It has been arranged the date for the next meeting, in which we will prepare the interview with a member of the maintenance staff will.
\end{enumerate}

\begin{tabular}{|p{5cm} c|p{5cm}|p{5cm}|}
\hline Activity & Person Responsible & DeadLine \\\hline
Research on the Internet about similar applications in order to get new ideas to our project. & Víctor de Juan & 09/02/2015\\\hline

Define the subsystems of the application. & Guillermo Julián Moreno & 09/02/2015\\\hline

Mandatory part of the research on the Internet about certain applications of websites whose functionalities are similar to the ones we have to offer. & Iván Márquez Pardo & 09/02/2015\\\hline
\end{tabular}


\subsection{Meeting Minutes - 5}
\textbf{Date and Time:} 09/02/2015, 14:00. 
\textbf{Length:} 24 minutes. 
\textbf{Participants: } Guillermo Julián Moreno, Víctor de Juan Sanz, Iván Márquez Pardo.

\textbf{Topics: } 
\begin{enumerate}
\item Revision of subsystems.
\item Preparation of the interview with the maintenance manager.
\end{enumerate}

\textbf{Decisions Made:}\\
\begin{enumerate}
\item Questions for the interview with the maintenance manager have been prepared.
\item Time to start preparing the presentation of our project proposal.
\end{enumerate}

\begin{tabular}{|p{5cm} c|p{5cm}|p{5cm}|}
\hline Activity & Person Responsible & DeadLine \\\hline
Prepare the presentation: Susbsystems and Requirements. & Víctor de Juan & 12/02/2015\\\hline

Prepare the presentation: Mock-ups of the application. & Guillermo Julián Moreno & 12/02/2015\\\hline

Prepare the presentation: Introduction, Project Definition, Scope, Conclusions. & Iván Márquez Pardo & 12/02/2015\\\hline
\end{tabular}


\subsection{Meeting Minutes - 6}
\textbf{Date and Time:} 11/02/2015, 11:35. 
\textbf{Length:} 51 minutes. 
\textbf{Participants: } Guillermo Julián Moreno, Víctor de Juan Sanz, Iván Márquez Pardo.

\textbf{Topics: } 
\begin{enumerate}
\item Overview of the interview.
\item Clarify some aspects of the interview.
\item Correction of some parts of the project.
\item Review of the presentation.
\item Practice the oral presentation.
\end{enumerate}

\textbf{Decisions Made:}\\
\begin{enumerate}
\item If we complete all tasks on time, only the reflection paper would remain undone.
\end{enumerate}

\begin{tabular}{|p{5cm} c|p{5cm}|p{5cm}|}
\hline Activity & Person Responsible & DeadLine \\\hline
Finish the Functional and Non-Functional requirements of the project. & Víctor de Juan & 20/02/2015\\\hline

Finish the mock-ups of the project. & Guillermo Julián Moreno & 17/02/2015\\\hline

Finish the project definition and conclusions. & Guillermo Julián Moreno & 20/02/2015\\\hline

Write down the clean-up version of the interview, research, meetings, introduction and scope of the project. & Iván Márquez Pardo & 12/02/2015\\\hline
\end{tabular}
\section{Meeting Announcements}
\subsection{Meeting Announcement - 1}

\textbf{From: } Iván Márquez Pardo

\textbf{To: } Guillermo Julián Moreno, Iván Márquez Pardo y Víctor de Juan Sanz.

\textbf{Date, Time and Place: } 27/01/2015 from 14:35 ultil we finish Brainstorming session in Science building, module 17.

\textbf{Purpose} Finish brainstorming session.

\subsection{Meeting Announcement - 2}


\textbf{From: }

\textbf{To: } Guillermo Julián Moreno, Iván Márquez Pardo y Víctor de Juan Sanz.


\textbf{Date, Time and Place: }


\textbf{Purpose: }


\textbf{Agenda: }

\textbf{Decision Follow-up: }


\textbf{Documentation: }


\section{Meetings Minutes}

We have excluded all in-class meetings. Should we add them?

\subsection{Meeting Minute - 1}


\textbf{Date and Time:} 27/01/2015 from 14:35 to 15:03.


\textbf{Participants: } Iván Márquez Pardo, Guillermo Julián Moreno y Víctor de Juan Sanz.


\textbf{Topics: }
Finish brainstorming and brainwriting session.


\textbf{Decitions Made: }
Víctor de Juan has to write clearly all the ideas in a document.

Brainstorm is finished, we have enough ideas to go to the next step.

\subsection{Meeting Minute - 2}


\textbf{Date and Time:}


\textbf{Participants: }

Iván Márquez Pardo, Guillermo Julián Moreno y Víctor de Juan Sanz.


\textbf{Topics: }


\textbf{Decitions Made: }


\end{document}
